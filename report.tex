\documentclass[UTF8]{article}
\usepackage{ctex}
\usepackage{listings}
\usepackage{xcolor}

\begin{document}
\author{隋唯一 2017011430}

\title{DPLL 实验报告}
\date{\today}
\maketitle

\section{实验目标}
实现一个简单的DPLL算法,对输入的cnf范式判定其可满足性,并给出一个使其为真的解释。
\section{实验思路}
DPLL算法基于深度优先搜索,所以总的时间复杂度是$O({2^n})$,关键是如何在细节上优化。\\
一个最直观的想法就是对所有的变量,逐一实验其所有的解释。比如,假定有三个变量$V_{1},V_{2},V_{3}$,先令其全部为true (当前指针指向$V_{3}$),检查cnf是否为真。若为真,返回true 以及当前的状态;若为假,指针暂不回退,指针指向的变量为false,再进行对cnf的检查。此时由于指针指向的变量true与false两个值都已经试验过了,所以当前变量(或说节点)置为“脏”,指针回退一步,指向$V_{2}$,并令$V_{2}$为false,$V_{3}$重新置为true。 如此循环,直到找到令cnf为真的解释,或所有可能都已经尝试。\\
然而这么做有一个问题,指针的回退不够智能,这种每次只回退一个节点,且直到所有节点都已经赋值才开始检查cnf的算法实际上等价于暴力遍历。比如,对于三个变量的情况,相当于遍历从000 到111的所有二进制数。\\
对于一个cnf,如果当前能够判断有一个子句是假,那么整个cnf都为假,指针就可以回退。对于cnf的一个子句,如果能判断一个literature为真,那这个子句就是真,可以跳出这个子句,进行对下一个子句的判断。也就是说,某些情况下,没有确定所有变量的取值,就可以判定cnf为假(此时指针回退),或cnf 为真(返回解)。\\
指针回退的目的地,应当是当前节点上溯,遇到的第一个还可以尝试的祖先节点(就是说,这个节点并没“脏”)。
\section{算法实现}
\subsection{存储结构}
\lstinputlisting [language=C++,firstline=12,lastline=20]{DPLL.cpp}
这里,每一个atom就是一个节点,每个节点有三个状态0,1,2,0代表这个节点被刚刚扩展,还没有尝试true或false;1代表这个节点已经尝试过false,如果在这个节点上再扩展应该把它置为true;2代表这个节点的true与false都已经试过了,都不能令cnf为真,应该回溯。使用atomStatus存储节点的状态。
\subsection{整体框架}
\lstinputlisting[language=C++,firstline=1,lastline=28]{skeleton.cpp}
这是一个典型的dfs框架,但是并没有维护一个二叉链表,而是用atomStatus维护搜索树。注意,指针回退到某一个变量时,这个变量之后的所有变量的atomStatus都置为0,因为此时它们的祖先发生了变化。
\subsection{对特定解释的判断}
\lstinputlisting[language=C++,firstline=33,lastline=65]{DPLL.cpp}
这里的判断思路就是二层循环,外循环是每一个子句,内循环是子句中的每一个literature。每一个内循环中,有一个literature为真,这个子句为真,循环结束。为循环中,有一个子句为假,cnf为假,外循环结束。最后的结果分为真、假、不确定。若为真,返回结果;若为假,指针回退;若不确定,指针加一,向下扩展。
\section{改进空间}
首先,针对存在不出现的atom的情况可以优化\\
其次,对于指针的回退,可以更加智能。
\section{实验心得}
通过本次实验,我对于dfs的理解更加深入,同时对搜索的使用更加灵活,不局限于二叉链表。

\end{document}
